\begin{definition}
Un \emph{problema de decisión} es aquel cuya únicas respuestas son ``SI" o ``NO".
Por ejemplo, los siguientes son problemas de decisión:
\begin{itemize}
    \item Dada una matriz cuadrada, decidir si es inversible.
    \item Dado un grafo y un natural $k$, decidir si es coloreable con $k$ colores. Este problema se suele llamar $k$-color.
\end{itemize}
\end{definition}

\begin{definition}
Llamamos \emph{P} a la clase de complejidad de los problemas de decisión para los cuales existe un algoritmo determinístico de tiempo polinomial en el tamaño de la entrada que lo resuelve.
\end{definition}

\begin{definition}
Llamamos \emph{NP} a la clase de complejidad de los problemas para los cuales existe un algoritmo no determinístico de tiempo polinomial en el tamaño de la entrada que lo resuelve.

Veremos luego que la siguiente es una definición equivalente: llamamos \emph{NP} a la clase de complejidad de los problemas para los cuales existe un algoritmo determinístico, que dado $X$ una posible solución para el $SI$, verifica en tiempo polinomial en el tamaño de $X$ si es o no una respuesta válida. 
\end{definition}

\begin{definition}
\emph{co-NP} es la clase de complejidad complementaria a \emph{NP}. Es decir son los problemas cuyas posibles respuestas por $NO$ son verificables en tiempo polinomial.
\end{definition}

\begin{proposition}
$P \subseteq NP \cap co-NP$
\end{proposition}
\begin{proof}
Obvio.
\end{proof}

\begin{conjecture}
$P = NP \cap co-NP$
\end{conjecture}

\begin{proposition}
$1$-color $\in P$.
\end{proposition}
\begin{proof}
Simplemente debemos determinar si $E = \varnothing$.
\end{proof}

\begin{proposition}
$2$-color $\in P$.
\end{proposition}

\begin{proof}
Como vimos antes: corriendo BFS desde algún vértice de cada componente conexa, asignamos como color a cada vértice la paridad de su nivel. Luego, verificamos si este coloreo es propio.\\
Este algoritmo es $\mathcal{O}(n+m) + \mathcal{O}(n+m)$.
\end{proof}

\begin{definition}[Reducción polinomial]
Una \emph{reducción en tiempo polinomial} o \emph{transformación polinomial} de un problema de decisión $A$ a un problema $B$ es un algoritmo de tiempo polinomial que transforma las entradas del problema A en entradas del problema B, tal que el problema transformado tiene las mismas salidas que el problema original.

Por lo tanto, una instancia $X$ de un problema A puede ser resuelto de la siguiente manera: aplicando esa transformación a $X$ obtenemos una instancia $Y$ del problema B y mediante un algoritmo para el problema B obtenemos una salida. Esta salida es a su vez solución de $Y$. 
Una reducción de este tipo se denota por $A \le_{p} B$.
\end{definition}

\begin{definition}[$NP-Completo$]
Un problema de decisión $C$ es \emph{$NP-completo$} si:
\begin{itemize}
    \item $C \in NP$
    \item Todo problema en $NP$ es reducible tiempo polinomial a $C$:
    $\forall D \in NP : D \le_{P} C$
\end{itemize}
\end{definition}

\begin{definition}
Un \emph{literal} es una variable o una negación de variable.

Una \emph{cláusula} es una disyunción o conjunción de literales.

Una \emph{fórmula} es una expresión booleana.

Una fórmula está en \emph{forma conjuntiva normal} o $CNF$ si es una conjunción de cláusulas disyuntivas.

Una fórmula está en \emph{forma disyuntiva normal} o $DNF$ si es una disyunción de cláusulas conjuntivas.
\end{definition}

\begin{definition}
Dada una fórmula $B$, el \emph{problema de satisfactibilidad booleana} o \emph{$SAT$} es un problema de decisión que consiste en determinar si existe una semántica (ie. una asignación de variables o una valuación) tal que $B$ sea verdadera 
\end{definition}

\begin{definition}
Dado $b \in \mathbb{Z}_2^n, B(b)$ es evaluar $B$ en $b$, considerar a $B$ como una función.
\end{definition}
\begin{definition}
$CNFSAT$ o $CNF-SAT$ es el mismo problema que $SAT$ pero con $B$ en $CNF$.\\
$DNFSAT$ o $DNF-SAT$ es el mismo problema que $SAT$ pero con $B$ en $DNF$.\\
$k-SAT$ es el mismo problema que $CNFSAT$ tal que cada clausula de $B$ contiene a lo sumo $k$ literales.
\end{definition}
\begin{proposition}
Ambos problemas se reducen a $SAT$:
\begin{itemize}
    \item $CNF-SAT \le_{P} SAT$
    \item $DNF-SAT \le_{P} SAT$
\end{itemize}
\end{proposition}
\begin{proof}
Obvio.
\end{proof}

\begin{theorem}
$DNF-SAT \in P$
\end{theorem}

\begin{theorem}
$2-CNF-SAT \in P$
\end{theorem}

\begin{theorem}
$CNF-SAT \le_P SAT$
\end{theorem}

\begin{theorem}
$CNF-SAT \le_P SAT$
\end{theorem}

\begin{theorem}
$CNF-SAT \le_P 3-CNF-SAT$
\end{theorem}
\begin{proof}
Cuando tenga tiempo.
\end{proof}

\begin{proposition}
$3-color \le_P 3-CNF-SAT$.
\end{proposition}
\begin{proof}
Debemos probar que dada B en $3-CNF-SAT$, podemos construir en tiempo polinomial un grafo $G$ tal que:
\begin{align}
    B \text{ es satisfacible} \iff \chi(G) = 3
\end{align}
Primero, una intuición:
\begin{itemize}
    \item Necesitaremos crear un grafo que dado un coloreo con 3 colores, retenga información sobre la semántica que debemos crear. Asimismo, dado una semántica que satisfaga $B$, debe haber una forma de plasmar esta información en un coloreo del grafo.
    \item Para lo anterior, debemos contar con una codificación de los valores posibles para una variable (1 o 0). Solo necesitaremos codificar 1, y lo codificaremos con un color especial.
    \item La estructura que usaremos para dado un coloreo extraer una semántica podría ser definir un vértice por literal en $B$. Esto no nos servirá para el otro problema: no siempre estos vértices estarán coloreados con el color especial. Así, definiendo no solo los literales en $B$ como vértices sino todos los literales posibles (llamaremos a estos vértices $v_{x_k}$s y $v_{\overline{x}_k}$s), y conectando un vértice de una variable con su negación y con un vértice extra, nos aseguramos que en este triángulo aparezcan los 3 colores. Es decir, el especial debe estar. Conectaremos otro vértice cuyo color tomaremos como aquel color especial (al que llamaremos $s$) con este vértice extra (al que llamaremos $t$), de modo que o bien una variable tiene el color especial o bien su negación lo tiene.
    \item Lo anterior se encarga de poder definir la semántica, pero debemos definirla en base a la estructura de $B$. Para esto, debemos incorporar también una serie de 3-uplas de vértices, que representen los 3 literales de cada disyunción (llamaremos a estos vértices $p$s). Al hacer esto, debemos procurar que cada uno de estos vértices este relacionado con el vértice que definimos antes (el $v_{x_k}$ o $v_{\overline{x}_k}$) correspondiente con su literal. No solo esto: para que cualquier 3-coloreo nos provea de una semántica que satisfaga $B$, debemos hacer que todo coloreo asigne colores a los $p$s de modo que alguno de los 3 correspondientes a cada disyunción haga que su $v_l$ asociado deba colorearse con el color especial, haciendo que la semántica lo haga verdadero y la disyunción sea verdadera.
    \item Para lograr lo anterior utilizamos de nuevo la estrategia de antes: conectando todos los $p$s con el vértice de color especial $s$, y conectándolos también con un triángulo (a los vértices de este triangulo los llamaremos $g$s), nos aseguramos que algún $g$ tenga el color de $t$, y como su $p$ asociado esta conectado con $s$, debe colorearse con el tercer color. Esto impacta en el $v_l$ asociado: sabemos que está conectado con $t$, así que solo le queda una opción: colorearse con el color especial (el de $s$). Así, sin importar el coloreo, siempre habrá un $p$ con color distinto al de $t$ y al de $s$, lo que hará que su $v_l$ tenga color $s$ y sea verdadero en la semántica.
\end{itemize}
Sean $x_1, x_2, \mathellipsis, x_n$ las variables en $B$, y  $B = (l_{11} \vee l_{12} \vee l_{13}) \wedge \mathellipsis \wedge (l_{m1} \vee l_{m2} \vee l_{m3})$.\\
Sean:
\begin{align}
    V &= \{ s, t \} \cup \{ v_{x_k}, v_{\overline{x}_k} \mid 1 \le k \le n \} \cup \{ p_{ij}, g_{ij} \mid 1 \le i \le m, 1 \le j \le 3 \}\\
    E &= \{ st \}
        \cup \{ tv_{x_k}, tv_{\overline{x}_k}, v_{x_k}v_{\overline{x}_k} \mid 1 \le k \le n \}\\
        &\cup \{ sp_{ij}, p_{ij}g_{ij}, p_{ij}v_{l_{ij}}
        \mid 1 \le i \le m, 1 \le j \le 3 \}\\
        &\cup \{ g_{i1}g_{i2}, g_{i2}g_{i3}, g_{i3}g_{i1} \mid 1 \le i \le m \}
\end{align}
Y sea $G = (V, E)$.\\
Podemos construir este grafo en tiempo polinomial: Solo debemos saber la cantidad de variables y la cantidad de términos.\\
\begin{definition}
Sean $b \in \mathbb{Z}_2^n$ y $C$ un coloreo propio de $G$. Diremos que $b$ y $C$ son compatibles si:
\begin{itemize}
    \item $x_k(b) = 1 \iff C(v_{x_k}) = C(s)$
    \item $x_k(b) = 0 \iff C(v_{\overline{x}_k}) = C(s)$
\end{itemize}
\end{definition}

\begin{proposition}
Si $b$ y $C$ son compatibles, $l(b) = 1 \iff C(v_l) = C(s)$.
\end{proposition}
\begin{proof}
Por casos:
\begin{itemize}
    \item Si $l = x_k$, $C(v_{x_k}) = C(s) \iff x_k(b) = 1 \iff l(b) = 1$
    \item Si $l = \overline{x}_k$, $C(v_{\overline{x}_k}) = C(s) \iff x_k(b) = 0 \iff \overline{x}_k(b) = 0 \iff l(b) = 1$
\end{itemize}
\end{proof}

\begin{enumerate}
    \item Caso 1: $B$ es satisfacible $\implies \chi(G) = 3$.\\
    Sea $b \in \mathbb{Z}_2^n$ tal que $B(b) = 1$. Colorearemos $G$ con 3 colores: $r,s,t$. Naturalmente los colores de $s$ y $t$ son sus ídems, y $r$ es el (r)estante.
    \begin{align}
    C(s) = s\\
    C(t) = t\\
    C(v_l) =
    \left\{
        \begin{array}{cc}
            s & l(b) = 1\\
            r & l(b) = 0\\
        \end{array}
    \right.\\
    \end{align}
    Es fácil ver que el lado $st$ no causa problemas. Tapoco lo causan los [triangulitos] $t, v_{x_k}, v_{\overline{x}_k}$, ya que cada uno tiene un color distinto.\\
    Además, estamos construyendo un coloreo compatible con $b$.\\
    Ahora, observemos que como $B(b) = 1$, para cada término de la conjunción existe al menos un literal $l_{ij}$ tal que $l_{ij}(b) = 1$. Sea $j_i$ el mínimo $j$ tal que $l_{ij}(b) = 1$.
    \begin{align}
    C(p_{ij}) =
    \left\{
        \begin{array}{cc}
            r & j = j_i\\
            t & j \neq j_i
        \end{array}
    \right.
    \end{align}
    Aquí también es fácil ver que los lados $sp_{ij}$ no causan problemas, y que los lados $v_lp_{ij}$ tampoco:
    \begin{itemize}
        \item Si $j = j_i$, $l_{ij}(b) = 1$ y como $b$ y $C$ son compatibles, $C(v_l) = s$, y no hay problemas pues $C(p_{ij}) \neq s$.
        \item Si $j \neq j_i$, $C(p_{ij}) = t$ y no hay problemas pues $C(v_{l_{ij}}) \neq t$..
    \end{itemize}
    Por último,
    \begin{align}
    C(g_{ij}) =
    \left\{
        \begin{array}{cc}
            t & j = j_i\\
            r,s & j \neq j_i \text{ y según corresponda}
        \end{array}
    \right.
    \end{align}
    Aquí coloreamos $g_{ij}$ con el color $t$, si $j = j_i$. A los otros dos $g_{ij}$'s, los coloreamos con $r$ a uno y con $s$ al otro. Es entonces claro que el [triangulito] $g_{i1}g_{i2},g_{i2}g_{i3},g_{i3}g_{i1}$ no causa problemas.\\
    Además, dado un lado $p_{ij}g_{ij}$:
    \begin{itemize}
        \item Si $j = j_i$, $C(p_{ij}) = r$, pero $C(g_{ij}) = t$.
        \item Si $j \neq j_i$, $C(p_{ij}) = t$, pero $C(g_{ij}) = r,s$.
    \end{itemize}
    Dimos un coloreo propio de $G$ con 3 colores. Es decir, $G$ es 3-coloreable.
    
    \item Caso 2: $\chi(G) = 3 \implies B$ es satisfacible.\\
    Dado un coloreo de G con 3 colores $C$, construyamos $b \in \mathbb{Z}_2^n$ tal que $B(b) = 1$. Llamémosle como antes a los colores: $s$ al de $s$, $t$ al de $t$ y $r$ al restante.\\
    Definamos $b$ de manera que
    \begin{align}
        x_k(b) =
        \left\{
        \begin{array}{cc}
        1 & \text{ si } C(v_{x_k}) = s\\
        0 & \text{ si } C(v_{x_k}) \neq s
        \end{array}
        \right.
    \end{align}
    Es fácil ver que $b$ y $C$ son compatibles:
    \begin{itemize}
        \item Por construcción, $C(v_{x_k}) = s \iff x_k(b) = 1$.
        \item Además, $C(v_{\overline{x}_k}) = s \implies C(v_{x_k}) \neq s$, ya que $v_{x_k}$ y $v_{\overline{x}_k}$ son vecinos. Entonces, $x_k(b) = 0$. Y por construcción, $x_k(b) = 0 \implies C(v_{x_k}) \neq s$, y como $v_{\overline{x}_k}$ es vecino de $t$ y de $v_{x_k}$, $C(v_{\overline{x}_k}) = s$.
    \end{itemize}
    También es fácil ver que no hay contradicciones:
    \begin{itemize}
        \item $C(t) \neq C(s) \wedge C(t) \neq C(v_l) \wedge C(v_{x_k}) \neq C(v_{\overline{x}_k}) \implies (C(v_{x_k}) = s \iff C(v_{\overline{x}_k}) \neq s)$. Así, si a una variable se le asigna 1, a su negación 0, y viceversa.
    \end{itemize}
    Ya definimos entonces $b$. Veamos ahora que $B(b) = 1$.\\
    
    Como en $G[g_{i1},g_{i2},g_{i3}]$ están presentes los 3 colores, debe existir un $j_i$ tal que $C(g_{ij_i} = t)$ para cada $i$. Por lo tanto, su vecino $p_{ij_i}$ tiene color $r$ (es decir, distinto al de $s$ y al de $t$) ya que es vecino de $s$ también. Vemos también que $p_{ij_i}$ es vecino de un $v_l$. Esto quiere decir que $l$ es el $j_i$-ésimo literal de la $i$-ésima disyunción. Como $C(p_{ij_i}) = r$ y $v_l$ es vecino de $t$, $C(v_l) = s$, y entonces $l(b) = 1$. Por lo tanto la $i$-ésima disyunción $D_i$ cumple $D_i(b) = 1$, y $B(b) = (D_1 \wedge)(b) \mathellipsis (\wedge D_m)(b) = 1$
    \end{enumerate}
\end{proof}

\begin{theorem}[Cook, 1973]
$CNF-SAT \in NP-Completo$
\end{theorem}

Entonces todo lo anterior (3-color, etc.) es $NP-Completo$.
