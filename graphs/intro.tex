\subsection{Introducción}

\begin{definition}
  Un \emph{grafo} (no dirigido) es un par $(V, E)$ donde $V$ es un conjunto y
  $E \subseteq \{\{x,y\} \mid x,y \in V \wedge x \neq y \}$.
\end{definition}

\begin{notation} Utilizaremos las siguientes convenciones:
  \begin{itemize}
  \item[] Los elementos de $V$ se llaman \emph{vértices} o \emph{nodos}.

  \item[] Los elementos de $E$ se llaman \emph{lados} o \emph{aristas}.

  \item[] El lado $\{x,y\}$ se denota $xy$ o $yx$.

  \item[] Se suele llamar $n$ a $|V|$ y $m$ a $|E|$.

  \item[] Denotamos $\Gamma(x) = \{y \in V \mid xy \in E \}$ al conjunto de
    \emph{vecinos} de $x$.

  \item[] Denotamos $d(x) = |\Gamma(x)|$ al \emph{grado} del vértice $x$.

  \item[] Denotamos $\delta = \min_{x \in V} d(x)$ al grado del nodo con el menor
    grado, y $\Delta = \max_{x \in V} d(x)$ al grado del nodo con mayor grado.
  \end{itemize}
\end{notation}

\begin{definition}
  Un grafo es \emph{regular} si $\delta = \Delta$, es decir todos los nodos
  tienen la misma cantidad de vecinos.
\end{definition}

\begin{definition}
  Un \emph{subgrafo} de $G = (V_G,E_G)$ es un grafo $H = (V_H,E_H)$ tal que 
  $V_H \subseteq V_G$ y $E_H \subseteq E_G$
\end{definition}

\begin{definition}
  Un \emph{camino} de $x_0$ a $x_t$ en un grafo $G = (V,E)$ es una sucesión de
  vértices $x_0, x_1, \mathellipsis, x_t$ tal que:
\begin{enumerate}
    \item $\forall 0 \le i < t \colon x_i x_{i+1} \in E$
    \item $\forall 0 \le i, j \le t \colon x_i \neq x_j$
\end{enumerate}
Definimos la \emph{relación} $\sim$ de la siguiente manera: $x \sim y$ si y
solo si existe un camino de $x$ a $y$.
\end{definition}

\begin{proposition}
  $x \sim y$ es una relación de equivalencia.
\end{proposition}
\begin{proof}
  Veamos que $\forall x,y,z \in V,$
  \begin{enumerate}
  \item $x \sim x$. La sucesión de $x$ es el camino vacío.
    
  \item $x \sim y \iff y \sim x$. Si $x_0, \mathellipsis, x_t$ es un camino
    de $x$ a $y$, entonces la sucesión $x_t, \mathellipsis, x_0$ es un camino
    de $y$ a $x$.
    
  \item $x \sim y \wedge y \sim z \iff x \sim z$. Si $x_0, \mathellipsis, x_t$
    es un camino de $x$ a $y$ y $x_{t}, \mathellipsis, x_{r}$ es un camino de
    $y$ a $z$, entonces es claro que la sucesión
    $x_0, \mathellipsis, x_t, \mathellipsis, x_r$ es un camino de $x$ a $z$.
  \end{enumerate}
\end{proof}

\begin{definition}
  La \emph{componente conexa} de $x$ es la clase de equivalencia a la que
  pertenece. Es decir, la componente conexa de $x$ es el subgrafo más grande
  $H$ de $G$, tal que $\forall v \in V_H \colon v \sim x$. Decimos que
  $G=(V,E)$ es un grafo \emph{conexo} si tiene una única componente conexa.
  Esto es, $\forall x,y \in V \colon x \sim y$.
\end{definition}

\begin{proposition}[Lema del apretón de manos]
  \begin{align}
    \sum_{x\in V} d(x) = 2m \label{handshaking_lemma}
  \end{align}
\end{proposition}

\begin{proof}
  Cada uno de los $m$ lado del grafo conecta dos vértices. Es decir, incrementa
  sus grados en 1. Se sigue que la suma de todos los grados es $2m$.
\end{proof}


\begin{definition}
  Vimos $BFS$ y $DFS$. Ambos algoritmos visitan todos los vértices de un
  grafo, pero quizá en distinto orden. Para recorrer un grafo, $DFS$ visita
  vértices hasta encontrar uno con todos sus vecinos visitados. Luego,
  retrocede un paso y repite esto. En cambio, $BFS$ visita todos los vecinos
  de un vértice, y luego hace esto con cada uno de los vecinos que visitó.
  
  En pseudo-python:

\begin{lstinputlisting}[language=python]{graphs/bfs.py}
\end{lstinputlisting}
\end{definition}
