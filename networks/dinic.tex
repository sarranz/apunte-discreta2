\subsection{Dinic-Even}

\begin{algorithm}
\caption{Algoritmo de Dinic-Even para encontrar flujo maximal}
\begin{algorithmic}
\Require network $N=(V,E,C)$
\Ensure  $F$ un flujo maximal
\Function{Dinic}{$network\ N$}
\State $F = 0$
\State $L$ = \Call{LayeredNetwork}{$N, F$} \Comment{ Construir un network auxiliar por niveles $L$ basado en $F$ y $N$}
\While{$t \notin L$}
    \State $F$ += \Call{BlockingFlow}{$L$}
    \State $L$ = \Call{LayeredNetwork}{$N, F$}
\EndWhile
\State \Return{$F$}
\EndFunction

\State

\Function{BlockingFlow}{layered network $L$}
\State $len = d_L(s,t)$
\State vertex path $P = [s]$
\State $G = 0$ \Comment{Este será flujo bloqueante}
    \While{$true$}
    \State $i = 0$
        \While{$i < len$}
            \State $v = P[i]$
            \If{$\Gamma^+(v) \neq \varnothing$}
                \Comment{Avanzar}
                \State $P$ += algún vecino de $v$
                \State $i$ += $1$
            \ElsIf{$v \neq s$}
                \Comment{No podemos llegar a $t$: Retroceder}
                \State $E_L$ -= $\ora{P[i-1] P[i]}$
                \State $i$ -= $1$
            \Else
                \State \Return{$G$} \Comment{Solo terminamos cuando $\Gamma^+(s) = \varnothing$}
            \EndIf
        \EndWhile
    
        \Comment{Llegamos a $t$: Incrementar}
        \State $\varepsilon = \min_{0\le i < len} \{ C(\ora{P[i] P[i+1]}) - G(\ora{P[i] P[i+1]})\}$
        \For {$i \in \{0, \mathellipsis, len-1\}$}
            \State $G(\ora{P[i] P[i+1]})$ += $\varepsilon$
            \If {$C(\ora{P[i] P[i+1]}) = G(\ora{P[i] P[i+1]})$}
                \State $E_L$ -= $\ora{P[i] P[i+1]}$
            \EndIf
        \EndFor
    \EndWhile
\EndFunction
\end{algorithmic}
\end{algorithm}

\begin{definition}[Network por niveles]
Un network $N = (V,E,C)$ es un \emph{network por niveles} si:
\begin{align}
 V &= \bigcup_{i=0}^r V_i \\
 E &= \{\ora{xy} \mid \forall i: 0 \le i \le r : x \in V_i \wedge y \in V_{i+1}\}
 \end{align}
 tal que $\forall i, j : 0 \le i < j \le r \wedge i \neq j : V_i \cap V_j = \varnothing$ \\
Y con $V_0 = \{s\} \text{ y } V_r = \{t\}$

Es decir, solo hay lados entre vértices de distintos niveles siendo el primer nivel el de $s$ y el último el de $t$.
\end{definition}

\begin{definition}[Network residual]
Dado un network $N = (V_N, E_N, C_N)$ y un flujo $F_N$, el \emph{network residual} $R$ = $R_F$ = $R_{F_N} = (V_R, E_R, C_R)$ es el network tal que:
\begin{itemize}
    \item $V_R = V_N$
    \item $(\ora{xz}) \in E_R \iff 
    \left\{ 
    \begin{array}{cc}
    \ora{xz} \in E_N \wedge F_N(\ora{xz}) < C_N(\ora{xz})     & \text{caso \textit{forward}} \\
    \text{o bien}  & \\
    \ora{zx} \in E_N \wedge F_N(\ora{zx}) > 0 & \text{caso \textit{backward}}
    \end{array}
    \right.$
    \item $C_R(\ora{xz}) = \left\{
    \begin{array}{cc}
        C_N(\ora{xz}) -
        F_N(\ora{xz})   & \text{caso \textit{forward}}\\
        F_N(\ora{zx})   & \text{caso \textit{backward}}
    \end{array}
    \right.
    $
\end{itemize}
\end{definition}

\begin{definition}[Network auxiliar por niveles o \textit{level graph}]
Dado un network residual $R = R_F = R_{F_N} = (V_R, E_R, C_R)$ definimos al \emph{network auxiliar por niveles} o $\boldsymbol{level}$  $\boldsymbol{graph}\ L = L_F = (V_L, E_L, C_L)$ de la siguiente forma:
\begin{itemize}
    \item $V_L = \{ x\in V_R \mid d_F(x) < d_F(t)\} \cup \{t\}$
    \item $E_L = \{\ora{xy} \in E_R \mid d_F(x) + 1 = d_F(y)$\}
    \item $C_L = C_R$
\end{itemize}

\end{definition}

\begin{definition}[Flujo bloqueante]
Un \emph{flujo bloqueante} o $\boldsymbol{blocking}$  $\boldsymbol{flow}$ es un flujo tal que todos los caminos dirigidos de $s$ a $t$ tienen al menos un lado saturado.
\end{definition}

\begin{algorithm}
\caption{Pseudocódigo de Algoritmos tipo Dinic}
\begin{algorithmic}
\State $F = 0$
\While{$d_F(t) < \infty$}
    \State Construir un \textit{level graph} $L = L_F$
    \State Hallar flujo bloqueante $G$ en $L$
    \State Modificar $F$ en $N$ de acuerdo a $G$
\EndWhile

\end{algorithmic}
\end{algorithm}

\begin{theorem}
Entre dos level graphs consecutivos la distancia entre $s$ y $t$ aumenta.
\end{theorem}

\begin{proof}
Sean $A$ y $B$ dos networks auxiliares consecutivos. Probaremos que $d_A(t) < d_B(t)$. 
Sabemos que $d_A(t) \le d_B(t)$ por el lema [\ref{distancias_no_decrecen_EK}] de Edmonds-Karp (seguimos utilizando el camino más corto).
Debemos entonces probar que $d_A(t) \neq d_B(t)$.\\
Primero, observemos que $s, x_1, \mathellipsis, x_{r-1}, t$ es un camino dirigido no saturado en $A \iff$ no lo es en $B$, puesto que si es no saturado en $A$ el flujo bloqueante satura uno de sus lados, y si no lo es en $B$ entonces no estaba en $A$ (el flujo bloqueante lo hubiera saturado).\\
Sea $s, x_1, \mathellipsis, x_{r-1}, t$ el mínimo camino dirigido no saturado en $B$. Como dijimos, este camino no estaba en $A$. Esto puede darse por dos razones:

\begin{enumerate}
    \item Caso 1: $\exists i : x_i \in V_B : x_i \notin V_A$: Falta un vértice $x_i$ en $A$ que si está en el $B$ (notar que $x_i \neq s,t$).\\
    Entonces $d_A(t) \le d_A(x_i)$, ya que si la distancia fuese menor hubiera estado en $A$. Por el lema de E-K [\ref{distancias_no_decrecen_EK}], $d_A(x_i) \le d_B(x_i)$.\\
    Así, $d_{A}(t) \le d_{A}(x_i) \le d_{B}(x_i) < d_{B}(t)$ ($x_i$ está antes de $t$ en el camino en $B$). Así, $d_A(t) < d_B(t)$.
    
    \item Caso 2: $\exists i : \ora{x_{i-1} x_i} \notin E_A$: falta un lado $\ora{x_{i-1} x_{i}}$ pero ningún vértice. 
    Esto puede pasar por dos razones (correspondientes a la definición de $E_R$ y $E_A$). \\
    Tomemos el mínimo $i$ tal que $\ora{x_{i-1} x_{i}} \notin E_A$:
    
    \begin{enumerate}
    \item Caso A: El lado está saturado o vacío (en $N$).\\
    Como en $E_B$ este lado si está, al incrementar el flujo en $A$ debimos usarlo en el otro sentido: con un camino $s, \mathellipsis, x_i, x_{i-1}, \mathellipsis, t$. Así,
    \begin{align}
        d_B(t)
        &= d_B(x_i) + b_B(x_i)\\
        &= d_B(x_{i-1}) + 1 + b_B(x_i)\\
        &\ge d_A(x_{i-1}) + 1 + b_A(x_i)\\
        &\ge d_A(x_{i-1}) + 1 + b_A(x_{i-1}) + 1\\
        &\ge d_A(t) + 2\\
        &> d_A(t)
    \end{align}
    Como en E-K.
    %$\left.
    %\begin{array}{cc}
    %(\ora{x_{i-1} x_{i}}) \notin  E_{R_{A}}     &  \\
    %(\ora{x_{i-1} x_{i}}) \in E_{R^*_{B}}     & 
    %\end{array}
    %\right\}$
    %solo puede ocurrir si $(\ora{x_{i} x_{i-1}}) \in E_L$\\
    %Veamos lo anterior:\\
    %Supongamos que 
    %$(\ora{x_{i+1} x_i}) \in E_L$, luego $d_A(x_{i+1}) + 1 = d_A(x_i)$\\
    %Entonces, $d_A(t) = d_A(x_i) + b_A(x_i)$. Pero $\ora{(x_i x_{i+1}})$ es el primer %lado que no está en $E_L$
    %\begin{align}
    %\forall k : 0 \le k < i &: (\ora{x_{k-1} x_{k}}) \in E_L \implies \\
    %\forall j : j < i &: d_A(x_j) = j 
    %\end{align}
    %Entonces, $(\ora{x_{i-1} x_{i}}) \notin E_{R_A} \wedge (\ora{x_{i-1} x_{i}}) \in E_{R^{*}_B}$
         
    \item Caso B: El lado no está saturado ni vacío: $d_A(x_i) + 1 \neq d_A(x_{i-1})$\\
    Esto significa que $d_A(x_{i}) \le d_A(x_{i-1}) + 1$ (si no, como el lado no esta saturado ni vacío, estaría en $E_A$).
    $\therefore d_A(x_i) < d_A(x_{i-1}) + 1$. Como el camino es mínimo en $B$,
    \begin{align}
        d_B(t)
        &= d_B(x_i) + b_B(x_i)\\
        &= d_B(x_{i-1}) + 1 + b_B(x_i)\\
        &\ge d_A(x_{i-1}) + 1 + b_A(x_i)\\
        &\ge d_A(x_i) + 1 + b_A(x_i)\\
        &\ge d_A(t) + 1\\
        &> d_A(t)
    \end{align}
    %Pero $d_A(x_{i-1}) = i-1$ (pues $\forall j < i : d_A(x_j) = j$\\
    %Por lo tanto, $d_A(x_{i}) \neq  i-1 +1 = i$, pero como $d_A(x_{i}) \le d_B(x_{i}) = i$, 
    %tenemos que $d_A(x_{i}) < i$\\
    %$\therefore d_A(t) = d_A(x_{i}) + b_A(x_{i}) < i + b_A(x_{i}) \le i + b_B(x_{i}) = d_B(x_{i}) + b_B(x_{i}) = d_B(t)$\\
    %$\therefore d_A(t) < d_B(t)$
    \end{enumerate}
    %Volviendo al caso $A$, tenemos que $d_A(x_i) + 1 = d_A(x_{i-1}) = i-1$\\
    %$\therefore d_A(x_i) = i - 2 < i$ y vale la prueba de 2B.
\end{enumerate}
\end{proof}

\begin{theorem}
La complejidad del paso de flujo bloqueante en el algoritmo de Dinic-Even es $\mathcal{O}(nm)$
\end{theorem}
\begin{proof}
Sean \begin{itemize}
    \item A := 'Avanzar'
    \item R := 'Retroceder'
    \item I := 'Incrementar $G$'
\end{itemize}
Una corrida se corresponde a:
$A^*(A^*R^*)^*I$.\\
Consideremos entonces las palabras de la forma $A^*x$ (con $x \in \{ R, I \}$).
Necesitamos saber cuántas palabras hay y cuál es la complejidad de cada una.
Tengamos en cuenta lo siguente:
\begin{itemize}
    \item $compl$(A) = $\mathcal{O}(1)$
    \item $compl$(R) = $\mathcal{O}(1)$
    \item $compl$(I) = $\mathcal{O}(n)$. 
\end{itemize}

Teniendo en cuenta que R borra un lado del \textit{level graph} y que I también borra al menos un lado (pues al incrementar el flujo se satura al menos un lado) tenemos que:
\begin{align}
    \#palabras(A^*x) \le m
\end{align}

Veamos ahora la complejidad de cada palabra:
\begin{align}
\underbrace{A\mathellipsis A}_j R &\rightarrow \underbrace{\mathcal{O}(1)+\mathellipsis+\mathcal{O}(1)}_j+ \mathcal{O}(1) = \mathcal{O}(j)\\
\underbrace{A\mathellipsis A}_j I &\rightarrow \mathcal{O}(j) + \mathcal{O}(n)
\end{align}
Estamos en un \textit{level graph} y A mueve el pivot $p$ desde $p$ hasta un vecino de $p$. Por lo tanto, A incrementa en 1 el nivel de $p$. Luego de a lo sumo $n$ Avanzar debo llegar a $t$ (o se produce un I o bien hay que Retroceder). Luego $j \le n$

\begin{align}
    \therefore compl(\text{A...AR}) &= \mathcal{O}(j) = \mathcal{O}(n)\\
    compl(\text{A...AI}) &= \mathcal{O}(j) + \mathcal{O}(n)\\
    &= \mathcal{O}(n) + \mathcal{O}(n)\\
    &= \mathcal{O}(n)
\end{align}

Conclusión: $\# palabras \in \mathcal{O}(m)\ \wedge \  compl(palabra) \in \mathcal{O}(n) \implies$
Complejidad del paso de flujo bloqueante del algoritmo de Dinic-Even es $\mathcal{O}(nm)$

\end{proof}

\begin{theorem}
La complejidad del algoritmo de Dinic-Even es $\mathcal{O}(n^2 m)$
\end{theorem}
\begin{proof}
En [ref] vimos que la distancia de $s$ a $t$ en level graphs consecutivos aumenta.\\
$\therefore$ Hay a lo sumo $n-1$ \textit{level graphs}.\\
Tambien vimos que la complejidad del paso bloqueante es $\mathcal{O}(nm)$.
\end{proof}

\begin{definition}
Un network $N = (V,E,C)$ se dice unitario si tiene capacidad $1$ en todos sus lados.\\
$\forall \ora{xy} \in E : C(\ora{xy}) = 1$ \\
$\forall x \neq s,t : |\Gamma^+(x)| = 1 \veebar |\Gamma^-(x)| = 1$
\end{definition}
\begin{theorem}
Dinic en un network unitario tiene complejidad $\mathcal{O}(m\sqrt{n})$
\end{theorem}

\begin{proof}
Supongamos que hago $\sqrt{n}$ flujos bloqueantes y el algoritmo aún no termina.
Sea $F$ el flujo que tenemos. Notemos que $F \in \mathbb{Z}$ y que $R_F$ es unitario:
\begin{itemize}
    \item Si $F < C$, $C' = 1$
    \item Si no, $C' = F = 1$
\end{itemize}
Sea $G$ el flujo maximal en $R_F$ obtenido por Dinic.
Como es flujo, $\forall x \neq s,t : out_G(x) = in_G(x)$
Como $C' \le 1$:
\begin{itemize}
    \item si $|\Gamma^-(x)| = 1, in_G(x) \le 1$, es decir $in_G(x) = 0$
\end{itemize}
\end{proof}

