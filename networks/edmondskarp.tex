\subsection{Edmonds-Karp}

\begin{lstinputlisting}[language=python]{networks/greedy.py}
\end{lstinputlisting}

\begin{definition}
  Sea $N = (V,E,C)$ un network y $F$ un flujo de $s$ a $t$, definimos a un
  \emph{camino $F$-aumentante} como una sucesión de vértices
  $s = x_0, \mathellipsis, x_r = t$ tales que para todo $0 \le i < r$, alguna
  de las siguientes se cumple:
  \begin{itemize}
  \item $\ora{x_i x_{i+1}} \in E$ y
    $F(\ora{x_i x_{i+1}}) < C(\ora{x_i x_{i+1}})$. Es decir, se puede enviar aún
    más flujo, pues el lado no está saturado. A estos lados los llamamos lados
    \emph{forward}.
  \item $\ora{x_{i+1} x_i} \in E$ y $F(\ora{x_{i+1} x_i}) > 0$. Es decir, se le
    puede \emph{''pedir``} flujo. A estos lados los llamamos lados
    \emph{backward}.
  \end{itemize}
\end{definition}

\begin{definition}
  Sea $N$ un network, $F$ un flujo, y $\{x_0,\mathellipsis,x_r\}$ un camino
  $F$-aumentante entre $s$ y $t$. Enviar $\varepsilon$ unidades de flujo a
  través de ese camino consiste en redefinir el flujo tal que:
  \begin{align}
    F(\ora{xy}) =
    \begin{cases}
      F(\ora{xy}) & |\ \text{$\ora{xy}$ no está en el camino aumentante}\\
      F(\ora{xy}) + \varepsilon & |\ \text{ $xy$ es forward}\\
      F(\ora{xy}) - \varepsilon & |\ \text{ $xy$ es backward}
    \end{cases}
  \end{align}
\end{definition}

\begin{definition}
  Sea $F$ un flujo en un network $N$ entre $s$ y $t$ y sea
  $x_0, \mathellipsis, x_r$ un camino $F$-aumentante.
  Definimos, para $0 \le i < r:$
  
  \begin{align}
    \varepsilon_i = 
    \begin{cases}
      C(\ora{x_ix_{i+1}}) - F(\ora{x_ix_{i+1}}) & |\ x_ix_{i+1} \text{ es forward}\\
      F(\ora{x_{i+1} x_i}) & |\ x_{i+1}x_i \text{ es backward}
    \end{cases}
  \end{align}
  y
  \begin{align}
    \varepsilon = \min_{0 \le i < r} \varepsilon_i
  \end{align}
\end{definition}

\begin{lemma}\label{flujo_camino_aumentante}
  Sea $F$ un flujo en un network $N$ entre $s$ y $t$ y sea
  $x_0, \mathellipsis, x_r$ un camino $F$-aumentante.
  Entonces, al enviar $\varepsilon$ unidades de flujo por
  este camino $F$-aumentante, el valor del flujo se incrementa en
  $\varepsilon$. Es decir, si $F^*$ es el flujo luego del cambio,

  \begin{align}
    V(F^*) = V(F) + \varepsilon
  \end{align}
\end{lemma}

\begin{proof}
  Es claro que la fuente sigue siendo productora y el resumidero consumidor.
  Probemos entonces \emph{feasibility} y \emph{conservación}:
  \begin{itemize}
  \item[] \emph{Feasibility}:
    \begin{enumerate}
    \item $0 \le F^*(\ora{xy})$
      \begin{itemize}
        \item[] Caso \emph{forward}: Es claro pues $0 \le F(\ora{xy}) \le F^*(\ora{xy})$.
        \item[] Caso \emph{backward}: Notemos
          \begin{align}
            F^*(\ora{x_{i+1} x_i})
            &= F(\ora{x_{i+1} x_i}) - \varepsilon\\
            &\ge F(\ora{x_{i+1} x_i}) - \varepsilon_i\\
            &\ge 0
          \end{align}
      \end{itemize}

    \item $F^*(\ora{xy}) \le C(\ora{xy})$
      \begin{itemize}
      \item[] Caso \emph{forward}: Notemos
        \begin{align}
          F^*(\ora{x_i x_{i+1}})
          &= F(\ora{x_i x_{i+1}}) + \varepsilon\\
          &\le F(\ora{x_i x_{i+1}}) + C(\ora{x_i x_{i+1}}) - F(\ora{x_i x_{i+1}})\\
          &\le C(\ora{x_i x_{i+1}})
        \end{align}
      \item[] Caso \emph{backward}: Es claro pues
          $F^{*}(\ora{xy}) \le F(\ora{xy}) \le C(\ora{xy})$.
      \end{itemize}
    \end{enumerate}

  \item[] \emph{Conservación}: Es claro que si $x$ no pertenece al camino, no
    pasa nada. Consideremos entonces para $x_i \neq s, t$, los lados antes y
    después de este vértice en el camino. Hay 4 casos posibles:
    \begin{enumerate}
    \item $x_{i-1}x_i$ y $x_ix_{i+1}$ son \emph{forward}. Es decir,
      $\ora{x_i x_{i+1}} \in E$ y $\ora{x_{i-1} x_i} \in E$. Vemos que
      \begin{align}
        F^*(\ora{x_{i-1} x_i}) = F(\ora{x_{i-1} x_i}) + \varepsilon
        &\implies in_{F*} (x_i) = in_F(x_i) + \varepsilon\\
        F^*(\ora{x_i x_{i+1}}) = F(\ora{x_i x_{i+1}}) + \varepsilon
        &\implies out_{F^*}(x_i) = out_F(x_i) + \varepsilon
      \end{align}
      Y entonces $out_{F^*} (x_i) - in_{F^*}(x_i) = 0$.

    \item $x_{i-1}x_i$ es \emph{backward} y $x_ix_{i+1}$ es \emph{forward}. Es decir,
    $\ora{x_i x_{i+1}} \in E$ y $\ora{x_i x_{i-1}} \in E$. Como
      \begin{align}
        F^*(\ora{x_i x_{i+1}}) &= F(\ora{x_i x_{i+1}}) + \varepsilon\\
        F^*(\ora{x_i x_{i-1}}) &= F(\ora{x_i x_{i-1}}) - \varepsilon
      \end{align}
      vemos que
      \begin{align}
        in_{F^*} (x_i) &= in_F(x_i)\\
        out_{F^*}(x_i) &= out_F(x_i) + \varepsilon - \varepsilon = out_F(x_i)
      \end{align}
      Y entonces $out_{F^*}(x_i) - in_{F^*}(x_i) = 0$.

    \item $x_ix_{i+1}$ es \emph{backward} y $x_{i-1}x_i$ es \emph{forward}. Es decir,
    $\ora{x_{i+1} x_i} \in E$ y $\ora{x_{i-1} x_i} \in E$. Como
    \begin{align}
        F^*(\ora{x_{i+1} x_i}) &= F(\ora{x_{i+1} x_i}) - \varepsilon\\
        F^*(\ora{x_{i-1} x_i}) &= F(\ora{x_{i-1} x_i}) + \varepsilon 
    \end{align}
    vemos que
    \begin{align}
        in_{F^*} (x_i) &= in_F(x_i) + \varepsilon - \varepsilon = in_F(x_i)\\
        out_{F^*}(x_i) &= out_F(x_i)
    \end{align}
    Y entonces $out_{F^*}(x_i) - in_{F^*}(x_i) = 0$.

    \item $x_{i-1}x_i$ y $x_ix_{i+1}$ son \emph{backward}. Es decir,
      $\ora{x_{i+1} x_i} \in E$ y $\ora{x_i x_{i-1}} \in E$. Vemos que
      \begin{align}
        F^*(\ora{x_i x_{i-1}}) = F(\ora{x_i x_{i-1}}) - \varepsilon
        &\implies out_{F^*}(x_i) = out_F(x_i) - \varepsilon\\
        F^*(\ora{x_{i+1} x_i}) = F(\ora{x_{i+1} x_i}) - \varepsilon
        &\implies in_{F*} (x_i) = in_F(x_i) - \varepsilon
      \end{align}
      Y entonces $out_{F^*} (x_i) - in_{F^*}(x_i) = 0$.
    \end{enumerate}
  \end{itemize}

  Y respecto a $V(F^*)$,
  \begin{itemize}
    \item[] Si $x_0x_1$ es \emph{forward}, es decir $sx_1 \in E$,
    \begin{align}
      V(F*)
      &= out_{F^*}(s) - in_{F^*}(s)\\
      &= (out_F(s) + \varepsilon) - in_{F}(s)\\
      &= V(F) + \varepsilon
    \end{align}

    \item[] Si $x_0x_1$ es \emph{backward}, es decir $x_1s \in E$,
    \begin{align}
      v(F^*)
      &= out_{F^*}(s) - in_{F^*}(s)\\
      &= out_F(s) - (in_F(s) - \varepsilon)\\
      &= V(F) + \varepsilon
    \end{align}
  \end{itemize}
\end{proof}

\begin{lemma}
  Sea $N$ un network, $F$ un flujo de $s$ a $t$ y $S$ un corte, entonces
  \begin{align}
    V(F) = F(S, S^c) - F(S^c, S)
  \end{align}
\end{lemma}

\begin{proof}
  Sea $x \in S$. Vemos que
  \begin{align}
    out_F(x) - in_F(x) =
    \begin{cases}
      0 & |\ x \neq s\\
      V(F) & |\ x = s
    \end{cases}
  \end{align}
  Y entonces
  \begin{align}
    V(F)
    &= \sum_{x \in S} out(x) - in(x)\\
    &= \sum_{\substack{x \in S \\ xy \in E}} F(x,y) - F(y,x)\\
    &= F(S,V) - F(V,S)\\
    &= F(S,S) + F(S, S^c) - F(S,S) - F(S^c, S)\\
    &= F(S,S^c) - F(S^c, S)
  \end{align}
\end{proof}

\begin{corollary}
  El valor de un flujo es una cota inferior de la capacidad de un corte.
  Esto es: $\forall F, S \colon V(F) \le cap(S)$.
\end{corollary}

\begin{proof}
  Por \emph{feasibility} y \emph{conservación} sabemos que
  $0 \le F(S^c, S)$, y entonces
  \begin{align}
    V(F)
    &= F(S, S^c) - F(S^c, S)\\
	  &\le F(S, S^c)\\
    &\le cap(S)
  \end{align}
\end{proof}

\begin{theorem}[\textit{Max-flow min-cut}]
  Sea $N = (V,E,C)$ un network, $F$ un flujo de $s$ a $t$ en $N$ y $S$ un corte minimal de $V$ entonces
  \begin{align}
    \underbrace{F \text{ es maximal}}_{A} \iff 
    \underbrace{\nexists \text{ caminos } F\text{-aumentantes entre $s$ y $t$}}_{B} \iff
    \underbrace{V(F) = mincutcap_S}_{C}
  \end{align}
\end{theorem}
\begin{proof}
  $C \implies A:$\\
  Asumamos $C)$. Sea $S$ un corte con $cap(S) = V(F)$\\
  Sea $G$ otro flujo. Tenemos por [ref] que
  \begin{align} V(G) \le cap(S) = V(F) \end{align}
  de lo que se sigue que $F$ es maximal.\\
  
  Notar que así mismo, $S$ es minimal pues dado un corte cualquiera $T$\begin{align}
    cap(T) \ge V(F) = cap(S)
  \end{align}
  $B \implies C:$\\
  Sea $S = \{s\} \cup \{x \in V \mid \exists \text{ camino $F$-aumentante de $s$ a $x$}$\}.
  Es claro que $t \notin S$ pues por hipótesis \\$\nexists$ caminos $F$-aumentantes de $s$ a $t$.\\
  $\therefore S$ es corte.\\
  Como $S$ es corte,
  \begin{align}
    V(F) = F(S, S^c) - F(S^c, S)
  \end{align}
  Sea $x \in S$, $y \in S^c$, $\ora{xy} \in E$. Buscamos $F(xy)$, y vemos que:
  \begin{itemize}
  \item $\exists$ un camino F-aumentante $(x_0, x_1,\mathellipsis, x_r)$ entre $s$ y $x$
  \item $\nexists$ un camino $F$-aumentante entre $s$ e $y$
  \end{itemize}
  Esto es, $(x_0, \mathellipsis, x_r, y)$ no es un camino $F$-aumentante.\\
  Como $(x_0, x_1, \mathellipsis, x_r)$ si lo es, y $\ora{xy} \in E$, concluimos que $\ora{xy}$ está saturado, ie. \\$F(\ora{xy}) = C(\ora{xy})$.
  \begin{align}
    \therefore \forall x,y: x \in S \wedge y \in S^c \wedge xy \in E : F(\ora{xy}) = C(\ora{xy})\\
    F(S, S^c) = \sum_{\substack{x\in S\\ y \notin S \\\ora{xy} \in E}} F(\ora{xy}) = \sum_{\substack{x\in S\\ y \notin S\\ \ora{xy} \in E}} C(\ora{xy}) = C(S, S^c) = cap(S)\\
  \end{align}
  Ahora, veamos el otro caso. Sea $x \in S^c, y \in S, \ora{xy} \in E$. Buscamos $F(\ora{xy})$. Con un análisis similar al anterior, teniendo en cuenta que ahora $x\in \Gamma^-(y)$, concluimos $xy$ está vacio ie. $F(\ora{xy}) = 0$.
  \begin{align}
    \therefore \forall x, y : x \in S^c \wedge y \in S \wedge xy \in E : F(\ora{xy}) = 0\\
    F(S^c, S) = \sum_{\substack{x\not\in S\\ y \in S\\ \ora{xy} \in E}} F(\ora{xy}) = \sum_{\substack{x\not\in S\\ y \in S\\ \ora{xy} \in E}} 0 = 0
  \end{align}
  Así,
  \begin{align}
    V(F) = F(S, S^c) - F(S^c, S) = cap(S) - 0 = cap(S) = mincutcap_S
  \end{align}
  Por último, $S$ es claramente minimal.
  
  $A \implies B$:\\
  Suponiendo $F$ maximal, supongamos que $B$ es falso, es decir, que existe un camino $F$-aumentante entre $s$ y $t$.\\
  Por el lema [\ref{flujo_camino_aumentante}], es posible enviar $\varepsilon$ unidades de flujo por ese camino y obtener un flujo $F^*$ con $V({F^*}) = V(F) + \varepsilon$. Esto es absurdo pues $F$ es maximal.
\end{proof}

\begin{algorithm}
  \caption{Algoritmo de Edmonds-Karp para hallar flujo maximal}
  \begin{algorithmic}
    \Require network $N=(V,E,C)$
    \Ensure  $(F, v, S)$ donde $F$ es flujo maximal, $v$ es el valor del flujo y $S$ es el corte minimal
    \Function{Edmonds-Karp}{$network\ N$}
    \State $F = 0, v = 0, \varepsilon(s) = \infty$
    \While{$true$} 
    \State $queue\ Q = qEmpty()$
    \State $qEnqueue(Q,s)$
    \State $S := \{s\}$
    \While{$qCnt(Q)$}
    \State $x = qDequeue(Q)$
    \State \ForAll{$z \in \Gamma^{+}(x) \cap S^c$}
    \If{$F(xz) < C(xz)$}
    \State $qEnqueue(Q,z)$
    \State $S = S \cup \{z\}$
    \State $a(z) = x;\ b(z) = 1$
    \State $\varepsilon(z) = min\{\varepsilon(x), C(xz) - F(xz)\}$
    \EndIf
    \EndFor
    \State \ForAll{$z\in \Gamma^{-}(x) \cap S^c$}
    \If{$F(zx) > 0$}
    \State $qEnqueue(Q,z)$
    \State $S = S \cup \{z\}$
    \State $a(z) = x;\ b(z) = -1$
    \State $\varepsilon(z) = min\{\varepsilon(x), F(zx)\}$
    \EndIf
    \EndFor
    \EndWhile
    \State
    \If{$t \in S$}
    \State $q = t;\ \varepsilon = \varepsilon(t);\ v = v+E$
    \While{$q \neq s$}
    \State $p = a(q)$
    \If{$b(q) == 1$}
    \State $F(pq) = F(pq) + \varepsilon$
    \State $F(qp) = F(qp) - \varepsilon$
    \EndIf
    \State q = p;
    \EndWhile
    \Else
    \State \Return{$F, v, S$}
    \EndIf
    \EndWhile
    \EndFunction
  \end{algorithmic}
\end{algorithm}

\begin{theorem}[Teorema de integralidad]\label{integrality}
  Sea $N=(V,E,C)$ un network tal que $\forall xy \in E : C(xy) \in \mathbb{Z}$ entonces existe flujo maximal entero.
  
  En otras palabras, en un network con capacidades enteras encontraremos siempre un flujo maximal entero.
\end{theorem}

\begin{proof}
  Lo probaremos por inducción, corriendo el algoritmo de Ford-Fulkerson en $n+1$ pasos, sean $F_0, \mathellipsis, F_n$ los sucesivos flujos obtenidos:\\
  Caso base: $F_0 = 0 \in \mathbb{Z}$.\\
  Caso inductivo:
  $F_k \in \mathbb{Z} \implies F_{k+1} \in \mathbb{Z}$\\
  $F_{k+1}$ se construye a partir de $F_k$, cambiando en algunos lados el valor $F_k$ por $F_k \pm \varepsilon$. Por lo tanto, si probamos que $\varepsilon \in \mathbb{Z}$, también tendremos que $F_{k+1} \in \mathbb{Z}$. Luego,
  \begin{align}
    \varepsilon = \min\{\epsilon_i\}
  \end{align}
  y \begin{center}
    $\epsilon_i = \begin{cases} 
        C(\ora{x_i x_{i+1}}) - F(\ora{x_i x_{i+1}}) & |\ \ora{x_i x_{i+1}} \text{ es lado forward}\\
        F(\ora{x_{i+1} x_i}) & |\ \ora{x_i x_{i+1}} \text{ es lado backward}
    \end{cases}$
  \end{center}
  Como las capacidades son todas enteras, $\epsilon_i$ es entero, $\varepsilon$  es entero y por lo tanto $F_{k+1}$.
  Esto prueba que si Ford-Fulkerson termina, obtenemos un flujo maximal entero.
  Como $v_{F_{k+1}} - v_{F_k} = \varepsilon \in \mathbb{Z} > 0$, por lo tanto $\varepsilon \ge 1$. Esto es, el ``salto" que se produce entre un flujo y el siguiente es discreto.
  Como existe una cota superior para el conjunto de valores de flujos posibles (por ej. $cap(S)$), esta sucesión de flujos debe ser finita. 
\end{proof}

\begin{definition}
  Dados $a,b\in V$ y un flujo $F$ en un network $N = (V,E,C)$. Definimos a la 
  \emph{distancia respecto de $F$ de $a$ hasta $b$} como:
  \begin{align}
    D_F(a,b) = 
    \begin{cases}
      0 & |\ a = b\\
      \#\{x_i x_{i+1} \in E\mid 0\le i < r \}  & |\ \{x_0, \mathellipsis, x_r\} \text{ es el menor camino $F$-aum. entre $a$ y $b$} \\ 
      \infty & |\ \nexists~ \text{camino $F$-aumentante entre $a$ y $b$}
    \end{cases}
  \end{align}
  Esto es, el numero de lados del menor camino aumentante relativo a $F$ entre $a$ y $b$.\\
  Asimismo, abusando de la notación definimos
  \begin{itemize}
  \item $d_F(x) = D_F(s,x)$
  \item $b_F(x) = D_F(x,t)$
  \end{itemize}
\end{definition}

\begin{lemma}\label{distancias_no_decrecen_EK}
  Sea $N=(V,E,C)$ un network y $F_1, F_2, \mathellipsis, F_\ell$ los flujos 
  parciales obtenidos por Edmonds-Karp (supongamos que $F_0 = 0$ es el flujo inicial).\\
  Abreviaremos
  \begin{align}
    d_k(x) \doteq d_{F_k}(x) \\
    b_k(x) \doteq d_{F_k}(x)
  \end{align} entonces tenemos que $\forall x \in V$:
  \begin{align}
    d_k(x) \le d_{k+1}(x) \\
    b_k(x) \le b_{k+1}(x)
  \end{align}
\end{lemma}

\begin{proof}
  $\forall i : 1 \le i \le \ell$ :\\
  \begin{enumerate}
  \item
    Sea $A_i \doteq \{x \in V \mid d_i(x) < d_{i-1}(x)\}$, queremos ver que 
    $A_i = \varnothing$ comenzando por notar que $s \notin A_i$ ya que $\forall i: d_i(s) = 0$.\\
    Supongamos que $\exists i : A_i \neq \varnothing$. Sea ese $i$ el mínimo posible. 
    A su vez, sea $x = \min\{d_i(v) \mid v \in A_i \}$, ie. tal que $d_i(x)$ sea mínimo. 
    Entonces, $\forall z : d_i(z) > d_i(x) \implies z\notin A_i$.\\
    Como $d_i(x) < d_{i-1}(x) \le \infty$ tenemos que $d_i(x) < \infty$ y debe 
    haber un camino $F_i$-aumentante de $s$ a $x$. 
    Sea $(x_0, \mathellipsis, x_r)$ el camino mínimo (es decir, con $r = d_i(x)$).\\
    Probemos ahora que $\nexists$ un camino $F_{i-1}$-aumentante de $s$ a $x$, 
    y luego lleguemos a una contradicción.\\
    Sea $z = x_{r-1}$. Como $(x_0,\mathellipsis, x_r)$ es el camino 
    $F_i$-aumentante de menor longitud entre $s$ y $x$ entonces 
    $(x_0, \mathellipsis, x_{r-1})$ es el camino $F_{i}$-aumentante de menor 
    longitud entre $s$ y $z$. Así, como $d_i(z) = r-1$ se sigue que $z \notin A_i$. 
    Concluimos que $d_{i-1}(z) \le d_i(z) < d_i(x) < d_{i-1}(x)$. \\
    $\therefore d_{i-1}(z) \le d_{i-1}(x)-2$.
    Pero consideremos que $d_{i-1}(x) \le d_{i-1}(z) + 1$, ya que se puede ir de $z$ a $x$.
    Como $d_{i-1}(z)+2 \le d_{i-1}(x)$, esto es absurdo.\\
    $\therefore \nexists$ camino $F_{i-1}$-aumentante $(s,\mathellipsis, z, x).$\\
    Ahora la contradicción:\\
    Como existe un camino $(s, \mathellipsis, z, x)$, debe pasar $x \in \Gamma^+(z)$ o bien $x \in \Gamma^-(z)$. La razón por la que no existe un camino $F_{i-1}$ aumentante puede ser alguna de las siguientes:
    \begin{itemize}
    \item $x \in \Gamma^+(z) \wedge F_{i-1}(\ora{zx}) = C(\ora{zx})$
    \item $x \in \Gamma^-(z) \wedge F_{i-1}(\ora{zx}) = 0$ 
    \end{itemize}
    Pero como el camino $F_i$-aumentante si existe,
    \begin{itemize}
    \item $x \in \Gamma^+(z) \wedge F_{i-1}(\ora{zx}) = C(\ora{zx}) \wedge F_i(\ora{zx}) < C(\ora{zx})$
    \item $x \in \Gamma^-(z) \wedge F_{i-1}(\ora{zx}) = 0 \wedge F_i(\ora{zx}) > 0$
    \end{itemize}
    
    La única forma en que esto puede ocurrir es que al pasar de $F_{i-1}$ a $F_i$ usemos un camino de la forma $(s, \mathellipsis, x, z)$, ya que:
    \begin{itemize}
    \item Si el último lado es backward, $x \in \Gamma^+(z) \wedge F_{i}(\ora{zx}) < F_{i-1}(\ora{zx})$ (primer razón).
    \item Si el último lado es forward, $x \in \Gamma^-(z) \wedge F_{i}(\ora{xz}) > F_{i-1}(\ora{xz})$ (segunda razón).
    \end{itemize}
    Como usamos Edmonds-Karp, este camino tiene longitud mínima.
    Entonces $d_{i-1}(z) = d_{i-1}(x) + 1$. Pero sabemos que $d_{i-1}(x) + 1  = d_{i-1}(z) \le d_{i-1}(x) - 2$. Esto es una contradicción. Concluimos:
    \begin{align}
      \nexists i : A_i \neq \varnothing
      \equiv
      \forall i : A_i = \varnothing
    \end{align}
    Es decir, $\forall i, x : x \in V : d_{i-1}(x) \le d_i(x)$
  \item
    Similarmente, podemos construir un $B_i = \{x \in V \mid b_i(x) < b_{i-1}(x) \wedge i \ge 1\}$ y probar que
    \begin{align}
      \forall i : B_i = \varnothing && \text{ie.}\\
      \forall i, x : x \in V : b_{i-1}(x) \le b_i(x)
    \end{align}
  \end{enumerate}
\end{proof}

\begin{theorem}[Complejidad del algoritmo de Edmonds-Karp] 
  El algoritmo de Edmonds-Karp para encontrar un flujo maximal y un corte minimal en un network es de complejidad $\mathcal{O}(n m^2)$
\end{theorem}
\begin{proof}
  Notemos que al construir cada $F_i$ pasa una de las siguientes:
  \begin{itemize}
  \item Al menos un lado forward se satura.
  \item Al menos un lado backward se vacía.
  \end{itemize}
  Llamamos \emph{crítico} a un tal lado.
  Supongamos que un lado con vértices $x$ y $z$ se volvió crítico en el paso $i$. Antes de que pueda volverse crítico deberá usarse para el otro lado (si se usó como forward, debe usarse como backward, y viceversa). Supongamos que sucede eso en el paso $j > i$.\\
  Así, $(s, \mathellipsis, z, x)$ es un camino mínimo en el paso $i$, y $(s,\mathellipsis, x, z)$ en el paso $j$.
  Como el camino $F$-aumentante de $s$ a $t$ en el paso $j$ contiene a $z$:
  \begin{align}
    d_j(t) = d_j(z)+b_j(z)
  \end{align}
  Además, $z$ está después de $x$ en este camino:
  \begin{align}
    d_j(t) = d_j(x) + 1 + b_j(z)
  \end{align}
  Por el lema [\ref{distancias_no_decrecen_EK}], sabemos que $d_{i-1}(x) \le d_i(x)$ (haciendo la sustitución: $i-1 := i \wedge i := j$) y:\begin{align}
    d_j(t) \ge d_i(x) + 1 + b_i(z)
  \end{align}
  Y ahora, en el paso $i$, $z$ está antes que $x$:
  \begin{align}
    d_j(t) \ge d_i(x) + 1 + 1 + b_i(x)
  \end{align}
  Y entonces:
  \begin{align}
    d_j(t) \ge d_i(t) + 2
  \end{align}
  Concluimos que si un lado se vuelve crítico, antes de que pueda volver a hacerlo, la distancia entre $s$ y $t$ debe aumentar en al menos $2$.\\
  Por lo tanto, un lado puede volverse crítico a lo sumo $\frac{n+1}{2} \in \mathcal{O}(n)$ veces.
  Sabemos que en la construcción de cada $F_i$ al menos un lado se vuelve crítico. Así, tenemos que \begin{align}
    \# F_i &\le m * \# \text{veces que puede volverse crítico}\\
    \# F_i &\in \mathcal{O}(nm)
  \end{align}
  Cada corrida consiste en construir un $F_i$ mediante $BFS \in \mathcal{O}(m)$.\\
  $\therefore$ La complejidad del algoritmo Edmonds-Karp es $\mathcal{O}(m) * \mathcal{O}(nm) = \mathcal{O}(nm^2)$
\end{proof}

