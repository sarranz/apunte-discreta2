\subsection{Introducción}

\begin{definition}
  Un \emph{network} o \emph{red de flujo}  N es un grafo dirigido con capacidades
  en los lados. Es decir, una $3$-upla $(V, E, C)$ donde $V$ es un conjunto,
  $E \subseteq V \times V$ y $C \colon E \to \R_{\ge 0}$.

  No consideraremos networks con lados $(x,x)$ o con lados $(x,y)$ y $(y,x)$.
\end{definition}

\begin{notation}

  Los elementos de $V$ se llaman \emph{vértices} o \emph{nodos}.

  Los elementos de $E$ se llaman \emph{lados} o \emph{aristas}.

  $C$ se dice función de \emph{capacidad}.

  Escribiremos $\ora{xy}$ para denotar el lado $(x,y)$.
\end{notation}

\begin{definition}
  Sea G = (V,E) un grafo dirigido, definimos:
  \begin{itemize}
  \item La \emph{vecindad hacia adelante} de $x$ como
    $\Gamma^{+}(x) = \{y\mid \ora{xy} \in E\}$.
  \item La \emph{vecindad hacia atrás} de $x$ como
    $\Gamma^{-}(x) = \{y\mid \ora{yx} \in E\}$.
  \end{itemize}
\end{definition}

\begin{definition}
  Sea G = (V,E) un grafo dirigido, un \emph{camino dirigido} entre $x_0$ y $x_r$
  es una sucesión de vértices $\{x_0,\mathellipsis, x_r\}$ tal que
  \begin{align}
    \forall 0 \le i < r \colon \ora{x_i,x_{i+1}} \in E \nonumber
  \end{align}
\end{definition}

\begin{definition}
  Sean $s,t \in V$, un \emph{flujo} de $s$ a $t$ en un network $N = (V, E, C)$
  es una función $F \colon E \to \mathbb{R}_{\ge 0}$ tal que se cumple:
  \begin{enumerate}
  \item Feasibility o restricción de capacidad:
    \begin{align}
      \forall \ora{xy} \in E \colon 0 \le F(\ora{xy}) \le C(\ora{xy}) \label{feasible}
    \end{align}

  \item Conservación:
    \begin{align}
      \forall x \neq s,t \colon \sum_{\substack{\ora{xy} \in E}} F\left(\ora{xy}\right)
      = \sum_{\substack{\ora{yx} \in E}} F\left(\ora{yx}\right) \label{conservacion}
    \end{align}

  \item La fuente ($s$) es productora:
    \begin{align}
      \sum_{x \in V} F(\ora{sx}) - \sum_{x \in V} F\left(\ora{xs}\right) \ge 0 \label{fuente}
    \end{align}

  \item El resumidero ($t$) es consumidor:
    \begin{align}
      \sum_{x \in V} F(\ora{xt}) - \sum_{x \in V} F(\ora{tx}) \ge 0 \label{resumidero}
    \end{align}
  \end{enumerate}
\end{definition}

\begin{definition}
Sean $A \subseteq V$, $B \subseteq V$ y $F$ un flujo, definimos:
\begin{align}
  F(A, B) = \sum_{\substack{x \in A \\ y \in B \\ \ora{xy} \in E}} F(\ora{xy}) \nonumber
\end{align}
\end{definition}

\begin{proposition}
  Sean un network $(V,E,C)$, conjuntos $A, B, C \subseteq V$ con $B$ y $C$ disjuntos,
  y $F$ un flujo. Entonces
  \begin{enumerate}
    \item $F(A, B \cup C) = F(A,B) + F(A,C)$
    \item $F(B \cup C, A) = F(B,A) + F(C,A)$   
  \end{enumerate}
\end{proposition}

\begin{proof}
  Veamos los dos casos:
  \begin{enumerate}
  \item
    \begin{align}
      F(A, B \cup C)
      &= \sum_{\substack{x \in A \\ y \in B \cup C \\ \ora{xy} \in E}} F(\ora{xy})\\
      &= \sum_{\substack{x \in A \\ y \in B \\ \ora{xy} \in E}} F(\ora{xy})
      + \sum_{\substack{x \in A \\ y \in C \\ \ora{xy} \in E}} F(\ora{xy}) && \text{pues son disjuntos}\\
      &= F(A,B) + F(A,C)
    \end{align}

  \item
    \begin{align}
      F(B \cup C, A)
      &= \sum_{\substack{x \in B \cup C \\ y \in A \\ \ora{xy} \in E}} F(\ora{xy})\\
      &= \sum_{\substack{x \in B \\ y \in A \\ \ora{xy} \in E}} F(\ora{xy})
      + \sum_{\substack{x \in C \\ y \in A \\ \ora{xy} \in E}} F(\ora{xy}) && \text{pues son disjuntos}\\
      &= F(B,A) + F(C,A)
    \end{align}
  \end{enumerate}
\end{proof}

\begin{definition}\label{in/out}
  Para un vértice $x \in V$, definimos
\begin{align}
  out_{F}(x) &= F(\{x\}, V)\\
  in_{F}(x) &= F(V, \{x\})
\end{align}
\end{definition}

Podemos usar esto y redefinir las propiedades:
\begin{itemize}
  \item[\ref{conservacion}] $\forall x \neq s,t \colon out(x) = in(x)$
  \item[\ref{fuente}] $out(s) - in(s) \ge 0$
  \item[\ref{resumidero}] $in(t) - out(t) \ge 0$
\end{itemize}

En muchos casos, ocurrirá que $in(s) = 0$ y $out(t) = 0$.

\begin{proposition}
Si $F$ es flujo de $s$ a $t$ en un network $(V, E, C)$, entonces
\begin{align}
    in(t) - out(t) = out(s) - in(s)
\end{align}
\end{proposition}

\begin{proof}
\begin{align}
  0
  &= F(V,V) - F(V,V)\\
  &= \sum_{x \in V} F(V, \{x\}) - \sum_{x\in V} F(\{x\}, V)\\
  &= \sum_{x \in V} F(V, \{x\}) - F(\{x\}, V)\\
  &= \sum_{x \in V} out(x) - in(x)\\
  &= \sum_{\substack{x \in V \\ x \neq s,t}} out(x) - in(x) + \sum_{\substack{x \in \{s,t\}}} out(x) - in(x)\\
  &= \sum_{\substack{x \in \{s,t\}}} out(x) - in(x) &&\text{por \ref{conservacion}}\\
  &= out(s) - in(s) + out(t) - in(t)
\end{align}
Concluimos,
\begin{align}
  in(t) - out(t) &= out(s) - in(s)
\end{align}
\end{proof}

\begin{definition}
  El \emph{valor del flujo} F de $s$ a $t$ se define $V(F) = out(s) - in(s)$.
\end{definition}

\begin{proposition}
  Dado un flujo $F$, $V(F) \ge 0$.
\end{proposition}
\begin{proof}
  Por \ref{fuente}: $V(F) = out(s) - in(s) \ge 0$.
\end{proof}

\begin{definition}
  Un flujo F es \emph{maximal} si su valor es maximal. Es decir, si
  $\forall G \text{ flujo} \colon V(G) \le V(F)$.
\end{definition}

\begin{definition}
  Sea $N = (V,E,C)$ un network, y $F$ un flujo. Un lado $\ora{xy} \in E$ se dice
  saturado si $F(\ora{xy})$ = $C(\ora{xy})$.
\end{definition}

\begin{definition}
  Dado un network $N = (V, E, C)$, un \emph{corte} $S$ es un subcojunto de $V$
  que contiene a $s$ y no contiene a $t$. Es decir, $S \subseteq V$,
  $s \in S$, y $t \not\in S$.
\end{definition}

\begin{definition}
  Sea $N = (V, E, C)$ un network. La \emph{capacidad} de un corte S se define:
  \begin{align}
    cap(S) = C(S, S^c) = \sum_{\substack{x\in S \\ y \not\in S \\ xy \in E}} C(\ora{xy})
  \end{align}
\end{definition}

\begin{definition}
  Un corte $S$ de un network $(V, E, C)$ se dice \emph{minimal} si su
  capacidad es minimal. Es decir, si
  $\forall T \text{ corte} \colon cap(S) \le cap(T)$.
\end{definition}
