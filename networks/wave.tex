\subsection{Wave}
En Edmonds-Karp y Dinic el invariante es que $F$ es flujo, el algoritmo transforma un flujo en otro, y se detiene cuando no hay mas caminos aumentantes (E-K) o no hay mas caminos bloqueantes (Dinic).
Varios algoritmos invierten el esquema: se parte de un pseudo-flujo bloqueante, se lo va transformando en otros pseudo-flujos bloqueantes, hasta obtener un flujo propiamente dicho.

Wave: como $MKM$ pero desde $s$.
Veamos el paso bloqueante:
Sea $(x_0 = s, x_1, \mathellipsis, x_r = t)$ es el orden $BFS$ de la construcción de este network auxiliar.
\begin{algorithm}
\begin{algorithmic}
\Function{Paso bloqueante de Wave}{network N}
\State $G  = 0$
\For{$x\in V$}
\State $D(x) = 0$ \Comment{desbalance in-out}
\EndFor
\For{$x \neq s \wedge x\in V$}
\State $D(x) = 0$
\EndFor
\For{$x \neq s \wedge x \in V$} 
\State $B(x) = false$ \Comment{vértice bloqueado}
\State $A(x) = \varnothing$ \Comment{vértices que le mandan}
\EndFor
\For{$x \in \Gamma^+(s)$}
\State $G(\ora{sx}) = C(\ora{sx})$
\State $D(x)$ += $C(\ora{sx})$
\State $D(s)$ -= $C(\ora{sx})$
\State $A(x) = \{s\}$
\EndFor
\While{$D(s) + D(t) \neq 0)$}
\For{$x \in [x_1,\mathellipsis,x_{r-1}]$} \Comment{forward wave}
\If{$D(x) > 0$}
\State{$\Call{forwardBalance}{x}$}
\EndIf
\EndFor
\For{$x \in [x_{r-1} \mathellipsis, x_1]$} \Comment{backward wave}
\If{$D(x) < 0 \wedge B(x)$}
\State{$\Call{backwardBalance}{x}$}
\EndIf
\EndFor
\EndWhile
\EndFunction
\end{algorithmic}
\end{algorithm}

\begin{algorithm}
\begin{algorithmic}
\Function{forwardBalance}{vertex x}
\While{$D(x) > 0 \wedge \Gamma^+(x) \neq \varnothing$}
\State elegimos $z \in \Gamma^+(x)$
\If{B(z)}
\State{$\Gamma^+(x) := \Gamma^+(x) - \{z\}$}
\Else
\State $m = min\{D(x), C(\ora{xz}) - G(\ora{xz})\}$
\State $G(\ora{xz}$ += $m$
\State $D(x)$ -= $m$
\State $D(z)$ += $m$
\State $A(z) = A(z) \cup \{x\}$
\If{$G(\ora{xz}) == C(\ora{xz})$}
\State $\Gamma^+(x) := \Gamma^+(x) - \{z\}$
\EndIf
\If{$D(x) > 0$}
\State $B(x) := true$
\EndIf
\EndIf
\EndWhile
\EndFunction
\end{algorithmic}
\end{algorithm}

\begin{algorithm}
\begin{algorithmic}
\Function{backwardBalance}{vertex x}
\While{$D(x) > 0$}
\State elegimos $z \in A(x)$
\State $m = min\{D(x), G(\ora{zx})\}$
\State $G(\ora{zx}$ -= $m$
\State $D(x)$ -= $m$
\State $D(z)$ -= $m$
\If{$G(\ora{zx}) == 0$}
\State $A(x) := A(x) - \{z\}$
\EndIf
\EndWhile
\EndFunction
\end{algorithmic}
\end{algorithm}

\begin{theorem}
La complejidad del algoritmo de Wave es $\mathcal{O}(n^3)$
\end{theorem}

\begin{proof}
Wave es de tipo Dinic. Tiene $\mathcal{O}(n)$ networks auxiliares. Probemos que el paso bloqueante es $\mathcal{O}(n^2)$
Al correr el paso bloqueante hacemos una serie de \Call{forwardBalance}{x} y \Call{backwardBalance}{x}.
Al hacer \Call{ForwardBalance}{x} procesamos lados $\ora{xz}$ de dos categorias posibles:
\begin{itemize}
    \item Categoría T(otal): al procesarlo, $G(\ora{xz}) = C(\ora{xz})$. Es decir, se satura o vacía el lado.
    \item Categoría P(arcial): al procesarlo, $G(\ora{xz}) < C(\ora{xz})$
\end{itemize}
En los T se remueve de $\Gamma^+(x)$. Solo se entra una vez a T.
El total de lados de categoria T sobre todos los \Call{forwardBalance} sobre todos las olas es $\mathcal{O}(m)$.
De los de categoría P hay a lo sumo uno en cada \Call{forwardBalance}{x}:

$\#$ lados en cat P $\le \#FB$ totales.\\
En una ola hacia adelante hay a lo sumo $(n-2)$ FB(x).
En una de estas olas, puede pasar:
\begin{itemize}
    \item Todos los vértices quedan balanceados.
    \item Queda alguno no balanceado.
\end{itemize}
Toda ola bloquea al menos un vértice.
Como los vértices nunca se desbloquean, hay $\mathcal{O}(n)$ olas hacia adelante.\\
Entonces:
\begin{align}
\#FB &\le (n-2) \# olas
     \le (n-2) \mathcal{O}(n) = \mathcal{O}(n^2)
\end{align}

\text{Complejidad de todos las olas hacia adelante} = \# lados tipo T + \# lados tipo P\\
= $\mathcal{O}(m) + \mathcal{O}(n^2) = \mathcal{O}(n^2)$

Olas hacia atrás:
En los BB(x) también hay dos categorias:\begin{itemize}
    \item T: $G(\ora{zx})  = 0$
    \item P: $G(\ora{zx}) > 0$
\end{itemize}
Hay a lo sumo uno de tipo $P$ en cada BB(x).\\
$\#$ tipo P $\le \#$ BB(x)
$\le \mathcal{O}(n) \cdot \#$ olas hacia atras \\
$\le \mathcal{O}(n) \cdot \#$ olas hacia adelante
$\le \mathcal{O}(n) \cdot \mathcal{O}(n)$\\

Tipo T:
Si $\ora{zx}$ es de tipo T, $z$ se remueve de $A(x)$
¿Cuántas veces puedo agregar a $z$ a $A(x)$ luego de esto?
Necesito que $z$ le mande flujo a $x$ para que pase.
Pero solo hacemos BB(x) si $x$ está bloqueado y si $x$ está bloqueado, como nunca se desbloquea, $z$ no le puede mandar flujo.\\
Así, hay a lo sumo $m$ lados de tipo T y $\#$ olas hacia atrás $\le \#$ tipo T $+ \#$ tipo P
$\le \mathcal{O}(m) + \mathcal{O}(n^2)
\le \mathcal{O}(n^2)$
\end{proof}

\begin{lstlisting}[language=Python]
# levels = [lvl_1, ...., lvl_r]
# where lvl_k contains vertices in the k-th level
# and lvl_0 = [s], lvl_r = [t]
def blockingFlow(AN):
    vs = concat(levels)
    D(s) = CAP({s})
    FB(s)
    while D(s) + D(t) is not 0:
        for v in vs if not B(v):
            FB(v)
        for v in reverse(vs) if B(v):
            BB(v)

def FB(v):
    for w in FNeighbours(v) if D(v) > 0:
        if B(w) or g(v,w) is c(v,w):
            continue
        m = min(D(v), c(v,w)-g(v,w))
        g(v,w) += m
        D(v) -= m
        D(w) += m
    if D(v) > 0:
        B(v) = true
        
def BB(v):
    for w in BNeighbours(v) if D(v) > 0:
        if g(w,v) is 0:
            continue
        m = min(D(v), g(w,v))
        g(w,v) -= m
        D(v) -= m
        D(w) += m
\end{lstlisting}
